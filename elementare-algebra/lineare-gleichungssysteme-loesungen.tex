\documentclass[a4paper]{article}
% generated by Docutils <http://docutils.sourceforge.net/>
\usepackage{fixltx2e} % LaTeX patches, \textsubscript
\usepackage{cmap} % fix search and cut-and-paste in Acrobat
\usepackage{ifthen}
\usepackage[T1]{fontenc}
\usepackage[utf8]{inputenc}
\usepackage{amsmath}
\usepackage{color}

%%% Custom LaTeX preamble
\usepackage{units, amsmath, amsfonts, amssymb, textcomp, gensymb, marvosym,wasysym}
\usepackage[left=2cm,right=2cm,top=1cm,bottom=1.5cm]{geometry}
\usepackage[version=3]{mhchem}
\usepackage{tocloft}
\usepackage{array}
\usepackage{fancyhdr}
\usepackage{fancybox,shadow}
\usepackage{setspace}
\fancypagestyle{myplain}
{
\fancyhf{}
\renewcommand\headrulewidth{0pt}
\renewcommand\footrulewidth{0pt}
\fancyfoot[C]{\thepage}
}
\pagestyle{myplain}
%\pagestyle{empty}
\setlength{\parskip}{\baselineskip} % Extra line between paragraphs
\setlength{\parindent}{0pt} % No indent at the start of paragraphs
\fontsize{12pt}{14.4pt}
\selectfont
\hyphenation{Win-kel-ge-schwin-dig-keit}

%%% User specified packages and stylesheets

%%% Fallback definitions for Docutils-specific commands

% admonition (specially marked topic)
\providecommand{\DUadmonition}[2][class-arg]{%
  % try \DUadmonition#1{#2}:
  \ifcsname DUadmonition#1\endcsname%
    \csname DUadmonition#1\endcsname{#2}%
  \else
    \begin{center}
      \fbox{\parbox{0.9\textwidth}{#2}}
    \end{center}
  \fi
}

% title for topics, admonitions, unsupported section levels, and sidebar
\providecommand*{\DUtitle}[2][class-arg]{%
  % call \DUtitle#1{#2} if it exists:
  \ifcsname DUtitle#1\endcsname%
    \csname DUtitle#1\endcsname{#2}%
  \else
    \smallskip\noindent\textbf{#2}\smallskip%
  \fi
}

% transition (break, fancybreak, anonymous section)
\providecommand*{\DUtransition}[1][class-arg]{%
  \hspace*{\fill}\hrulefill\hspace*{\fill}
  \vskip 0.5\baselineskip
}

% hyperlinks:
\ifthenelse{\isundefined{\hypersetup}}{
  \usepackage[colorlinks=true,linkcolor=blue,urlcolor=blue]{hyperref}
  \urlstyle{same} % normal text font (alternatives: tt, rm, sf)
}{}
\hypersetup{
  pdftitle={Lineare Gleichungssysteme},
}

%%% Title Data
\title{\phantomsection%
  Lineare Gleichungssysteme%
  \label{lineare-gleichungssysteme}%
  \label{losungen-lineare-gleichungssysteme}}
\author{}
\date{}

%%% Body
\begin{document}
\maketitle

% {{{

Die folgenden Lösungen beziehen sich auf die %
\raisebox{1em}{\hypertarget{id2}{}}\hyperlink{id1}{\textbf{\color{red}:ref:`Übungsaufgaben <Aufgaben
Lineare Gleichungssysteme>`}} zum Abschnitt %
\raisebox{1em}{\hypertarget{id4}{}}\hyperlink{id3}{\textbf{\color{red}:ref:`Lineare Gleichungssysteme
<Lineare Gleichungssysteme>`}}.

\DUadmonition[system-message]{
\DUtitle[system-message]{system-message}
\raisebox{1em}{\hypertarget{id1}{}}

{\color{red}ERROR/3} in \texttt{lineare-gleichungssysteme-loesungen.rst}, line~9

\hyperlink{id2}{
Unknown interpreted text role \textquotedbl{}ref\textquotedbl{}.
}}

\DUadmonition[system-message]{
\DUtitle[system-message]{system-message}
\raisebox{1em}{\hypertarget{id3}{}}

{\color{red}ERROR/3} in \texttt{lineare-gleichungssysteme-loesungen.rst}, line~9

\hyperlink{id4}{
Unknown interpreted text role \textquotedbl{}ref\textquotedbl{}.
}}


%___________________________________________________________________________
\DUtransition

%
\begin{itemize}

\item Multipliziert man die zweite Gleichung $(II)$ mit dem Faktor $2$,
so nehmen die Koeffizienten in der $x_1$-Spalte die gleichen Werte an:
%
\begin{align*}
(\mathrm{I}): \quad 4 \cdot x_1 + 2 \cdot x_2 &= -6 \\
(\mathrm{II}): \quad 2 \cdot x_1 - 3 \cdot x_2 &= -7
\end{align*}%
\begin{align*}
\Rightarrow (\mathrm{I}): \quad 4 \cdot x_1 + 2 \cdot x_2 &= -6 \\
(2 \cdot \mathrm{II}): \quad 4 \cdot x_1 - 6 \cdot x_2 &= -14
\end{align*}
Subtrahiert man nun beide Gleichungen voneinander, so bleibt die erste Zeile
unverändert, während die zweite Zeile durch die Differenz aus der ersten und
zweiten Gleichung ersetzt wird.
%
\begin{align*}
\Rightarrow (\mathrm{I}): \quad 4 \cdot x_1 + 2 \cdot x_2 &= -6 \\
(\mathrm{I} - \mathrm{II}): \qquad \;\,\, 0 + 8 \cdot x_2 &= +8
\end{align*}
Die zweite Zeile stellt nun eine Gleichung mit nur \emph{einer} Unbekannten dar;
beim Auflösen dieser Gleichung erhält man das Ergebnis $x_2 = 1$. Setzt
man diesen Wert für $x_2$ in die Gleichung $\mathrm{I}$ ein, so
erhält man für die andere Unbekannte:
%
\begin{equation*}
4 \cdot x_1 + 2 \cdot 2 = -6 \quad \Longleftrightarrow \quad 4 \cdot x_1 =
-8 \quad \Longleftrightarrow \quad x_1 = -2
\end{equation*}
Das Gleichungssystem hat somit die Lösung $\mathbb{L} = \{ (-2 ;\; 1) \}$.

%
\raisebox{1em}{\hypertarget{id6}{}}\hyperlink{id5}{\textbf{\color{red}:ref:`Zurück zur Aufgabe <lgs01>`}}

\DUadmonition[system-message]{
\DUtitle[system-message]{system-message}
\raisebox{1em}{\hypertarget{id5}{}}

{\color{red}ERROR/3} in \texttt{lineare-gleichungssysteme-loesungen.rst}, line~51

\hyperlink{id6}{
Unknown interpreted text role \textquotedbl{}ref\textquotedbl{}.
}}

\end{itemize}

% sy.solve( [ sy.Eq( 4*x + 2*y, -6 ), sy.Eq( 2*x - 3 *y, -7) ] )

% {y: 1, x: -2}


%___________________________________________________________________________
\DUtransition

%
\begin{itemize}

\item Bezeichnet man die Anzahl an Sätzen, die der erste Spieler gewonnen hat, mit
$x_1$ und entsprechend die Anzahl der vom anderen Spieler gewonnenen
Sätze mit $x_2$, so entspricht das Rätsel folgendem linearen
Gleichungssystem:
%
\begin{align*}
x_1 + 1 &= 2 \cdot (x_2 - 1) \\
x_1 - 1 &= x_2 + 1 \\
\end{align*}
Dieses Gleichungssystem kann beispielsweise dadurch gelöst werden, indem man
die zweite Gleichung nach $x_1$ auflöst; man erhält dadurch $x_1 =
x_2 + 2$. Setzt man diesen Ausdruck für $x_2$ in die erste Gleichung
ein, so erhält man:
%
\begin{align*}
(x_2+2) + 1 &= 2 \cdot (x_2 - 1) \\
x_2 + 3 &= 2  \cdot x_2 - 2) \\
x_2 &= 5 \\
\end{align*}
Der zweite Spieler hat somit insgesamt $5$ Sätze gewonnen, der erste
wegen der Beziehung $x_1 = x_2 + 2$ insgesamt $7$ Sätze.

%
\raisebox{1em}{\hypertarget{id8}{}}\hyperlink{id7}{\textbf{\color{red}:ref:`Zurück zur Aufgabe <lgs02>`}}

\DUadmonition[system-message]{
\DUtitle[system-message]{system-message}
\raisebox{1em}{\hypertarget{id7}{}}

{\color{red}ERROR/3} in \texttt{lineare-gleichungssysteme-loesungen.rst}, line~84

\hyperlink{id8}{
Unknown interpreted text role \textquotedbl{}ref\textquotedbl{}.
}}

\end{itemize}


%___________________________________________________________________________
\DUtransition


% }}}

\end{document}
